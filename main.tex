\documentclass{beamer}
\usepackage[utf8]{inputenc}
\usepackage{textpos} 
\usepackage{graphicx} 
\usepackage{wrapfig}

\usetheme{Boadilla}
\usecolortheme{default}



\title[Rapport Projet] 
{Rapport Projet sur la conception d'une application des gestion de projets informatiques}

\subtitle{Conception et élaboration d'un modèle suivant MERISE 2}

\author[HANFAOUI.K et KAFIF.I] 
{HANFAOUI Karim\inst{1} \and KAFIF. Imane\inst{2}}

\institute[EMSI] 
{
  \inst{1} EMSI, 3ème Année INFO G9\\
  \inst{2} EMSI, 3ème Année INFO G9
}

\date[DEC 2024] 
{EMSI Les Orangers, Décembre 2024}



\AtBeginSection[]
{
  \begin{frame}
    \frametitle{Table of Contents}
    \tableofcontents[currentsection]
  \end{frame}
}

\begin{document}



\begin{frame}
    \titlepage
    \begin{textblock*}{2cm}(9cm, -7.7cm) 
        \includegraphics[height=0.7cm]{logo} 
    \end{textblock*}
\end{frame}



\begin{frame}
\frametitle{Table of Contents}
\tableofcontents
\end{frame}

\begin{frame}
    \frametitle{List of Figures}
    \begin{itemize}
        \item \hyperlink{fig1}{Figure 1: Example Image}
        \item \hyperlink{fig2}{Figure 2: Another Example}
        \item \hyperlink{fig3}{Figure 3: Yet Another Example}
    \end{itemize}
\end{frame}


\section{Problématique}
\begin{frame}{C'est quoi notre Problématique}
    \begin{block}{Problématique}
        Comment concevoir et développer une application de gestion de projets informatiques capable de répondre efficacement aux besoins des équipes de développement en termes de suivi de l'avancement des projets, de gestion optimale des ressources, et de facilitation de la communication et de la collaboration, tout en s'adaptant aux contraintes organisationnelles et technologiques variées des entreprises ?
    \end{block}
\end{frame}

\section{Acronymes et Terminologie}
\begin{frame}{Acronymes et Terminologie}
    \textbf{MERISE} : Méthode d'Étude et de Réalisation Informatique pour les Systèmes d'Entreprise \\
    \textbf{MCD} : Modèle Conceptuel de Données \\
    \textbf{MLD} : Modèle Logique de Données \\
    \textbf{MCT} : Modèle Conceptuel de Traitement \\
    \textbf{MOT} : Modèle Organisationnel des Traitements \\
    \textbf{MCTA} : Modèle Conceptuel des Traitements Automatisés \\
    \textbf{MOTA} : Modèle Organisationnel des Traitements Automatisés \\
    \textbf{UML} : Unified Modeling Language (Langage de Modélisation Unifié) \\
    \textbf{USE CASE} : Cas d’Utilisation, outils de l'UML \\
    \textbf{Réseau de Petri} : Modèle mathématique pour la modélisation et l’analyse des processus concurrents \\
\end{frame}


\section{Résumé du cours}
\begin{frame}{Résumé complet du cours de ce semestre}
Voici le résumé du cours ;)
\end{frame}


\begin{frame}
    \frametitle{Figure 1: Example Image}
    \label{fig1}
    \begin{figure}
        \includegraphics[width=0.8\textwidth]{logo}
        \caption{This is an example image.}
    \end{figure}
\end{frame}

\begin{frame}
    \frametitle{Figure 2: Another Example}
    \label{fig2}
    \begin{figure}
        \includegraphics[width=0.8\textwidth]{logo}
        \caption{This is another example image.}
    \end{figure}
\end{frame}

\begin{frame}
    \frametitle{Figure 3: Yet Another Example}
    \label{fig3}
    \begin{figure}
        \includegraphics[width=0.8\textwidth]{logo}
        \caption{Yet another example image.}
    \end{figure}
\end{frame}


\section{Comparaison entre Merise 1 et Merise 2}

\begin{frame}{Qu'est-ce que MERISE 1 ?}
\begin{itemize}
    \item MERISE 1 est une méthodologie française d’analyse et de conception des systèmes d’information.
    \item Développée dans les années 1970, elle est utilisée pour modéliser et réaliser des bases de données et des applications.
\end{itemize}
\end{frame}

\begin{frame}{Principes fondamentaux de MERISE 1}
\begin{enumerate}
    \item \textbf{Séparation des niveaux de modélisation :}
    \begin{itemize}
        \item Modèle conceptuel (niveau conceptuel)
        \item Modèle logique (niveau organisationnel)
        \item Modèle physique (niveau technique)
    \end{itemize}
    \item \textbf{Cycle de vie du système d’information :}
    \begin{itemize}
        \item Étude préalable, étude détaillée, réalisation, exploitation et maintenance.
    \end{itemize}
    \item \textbf{Focus sur les données :}
    \begin{itemize}
        \item Modèle Conceptuel des Données (MCD)
        \item Modèle Logique des Données (MLD)
        \item Modèle Physique des Données (MPD)
    \end{itemize}
\end{enumerate}
\end{frame}

\begin{frame}{Caractéristiques de MERISE 1}
\begin{itemize}
    \item Approche systématique, adaptée aux projets bien définis et peu évolutifs.
    \item Centré sur les bases de données relationnelles.
    \item Modélisation utilisant des diagrammes (MCD, MCT, DFD).
\end{itemize}
\end{frame}

\begin{frame}{Limites de MERISE 1}
\begin{itemize}
    \item Peu adapté aux systèmes modernes (orientés objet, dynamiques).
    \item Approche rigide, difficilement compatible avec des méthodologies agiles.
\end{itemize}
\end{frame}

\begin{frame}{Conclusion}
\begin{itemize}
    \item MERISE 1 est efficace pour les systèmes d’information classiques.
    \item Son usage diminue avec l'émergence de méthodologies modernes comme UML.
\end{itemize}
\end{frame}
\section{Présentation du modèle conceptuel du traitement analytique}
\begin{frame}{Modèle conceptuel}
    Description du modèle conceptuel...
\end{frame}



\section{Conception du projet selon les règles de Merise 2}
\begin{frame}{Conception selon Merise 2}
    Explication des règles et de leur application au projet.
\end{frame}

\section{Bibliographies et Références}
\begin{frame}{Bibliographies}
\begin{thebibliography}{9}

\bibitem{bts_doc}
BTS CGO 2A P10 -\textit{ Organisation du Système d’Informations Fiche MCT, MCTA, MOT et MOTA.} Editions d'Organisation. 
\href{https://example.com/merise}{[Lien]} 

\bibitem{merise_miseaniveau}
Pr. S.EL MOUMNI -\textit{ Conception des systèmes d’information.} Cours mise à niveau. 
\alert{[Lien non disponible]}

\end{thebibliography}
\end{frame}

\begin{frame}
\frametitle{Sample frame title}

In this slide, some important text will be
\alert{highlighted} because it's important.
Please, don't abuse it.

\begin{block}{Remark}
Sample text
\end{block}

\begin{alertblock}{Important theorem}
Sample text in red box
\end{alertblock}

\begin{examples}
Sample text in green box. The title of the block is ``Examples".
\end{examples}
\end{frame}



\begin{frame}
\frametitle{Two-column slide}

\begin{columns}

\column{0.5\textwidth}
This is a text in first column.
$$E=mc^2$$
\begin{itemize}
\item First item
\item Second item
\end{itemize}

\column{0.5\textwidth}
This text will be in the second column
and on a second thought, this is a nice-looking
layout in some cases.
\end{columns}
\end{frame}


\end{document}
