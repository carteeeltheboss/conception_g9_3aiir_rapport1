\documentclass{beamer}
\usepackage[utf8]{inputenc}
\usepackage{textpos} 
\usepackage{graphicx} 
\usepackage{wrapfig}

\usetheme{Boadilla}
\usecolortheme{default}



\title[Rapport Projet] 
{Rapport Projet sur <x>}

\subtitle{Conception et élaboration d'un modèle suivant MERISE 2}

\author[HANFAOUI, Karim et J. Doe] 
{HANFAOUI Karim\inst{1} \and J. Doe\inst{2}}

\institute[EMSI] 
{
  \inst{1} EMSI, 3ème Année INFO G9\\
  \inst{2} EMSI, 3ème Année INFO G9
}

\date[DEC 2024] 
{EMSI Les Orangers, Décembre 2024}



\AtBeginSection[]
{
  \begin{frame}
    \frametitle{Table of Contents}
    \tableofcontents[currentsection]
  \end{frame}
}

\begin{document}



\begin{frame}
    \titlepage
    \begin{textblock*}{2cm}(9cm, -7.7cm) 
        \includegraphics[height=0.7cm]{logo} 
    \end{textblock*}
\end{frame}



\begin{frame}
\frametitle{Table of Contents}
\tableofcontents
\end{frame}



\section{Problématique}
\begin{frame}{C'est quoi notre Problématique}
    \begin{block}{Problématique}
        La problematique la !
    \end{block}
\end{frame}

\section{Résumé du cours}
\begin{frame}{Résumé complet du cours de ce semestre}
Voici le résumé du cours ;)
\end{frame}



\section{Comparaison entre Merise 1 et Merise 2}
\begin{frame}{Introduction}
    \begin{itemize}
        \item Merise est une méthode d'analyse, de conception et de gestion de projets informatiques.
        \item Merise 1 et Merise 2 sont deux itérations de cette méthode, avec des objectifs et des approches distinctes.
        \item Cette présentation explore leurs différences principales.
    \end{itemize}
\end{frame}



\section{Présentation du modèle conceptuel du traitement analytique}
\begin{frame}{Modèle conceptuel}
    Description du modèle conceptuel...
\end{frame}



\section{Conception du projet selon les règles de Merise 2}
\begin{frame}{Conception selon Merise 2}
    Explication des règles et de leur application au projet.
\end{frame}



\begin{frame}
\frametitle{Sample frame title}

In this slide, some important text will be
\alert{highlighted} because it's important.
Please, don't abuse it.

\begin{block}{Remark}
Sample text
\end{block}

\begin{alertblock}{Important theorem}
Sample text in red box
\end{alertblock}

\begin{examples}
Sample text in green box. The title of the block is ``Examples".
\end{examples}
\end{frame}



\begin{frame}
\frametitle{Two-column slide}

\begin{columns}

\column{0.5\textwidth}
This is a text in first column.
$$E=mc^2$$
\begin{itemize}
\item First item
\item Second item
\end{itemize}

\column{0.5\textwidth}
This text will be in the second column
and on a second thought, this is a nice-looking
layout in some cases.
\end{columns}
\end{frame}


\end{document}
